\section{Beispiele}

In diesem Abschnitt werden verschiedene Berechnungsbeispiele aufgeführt.


\subsection{Berechnung der Entropie einer Quelle mit Gedächtnis}

Eine Quelle mit GEdächtnis kann mit einem Markoff-Diagramm dargestellt werden.
Diese Überlegungen und Berechnungen gelten auch für das Kanalmodell.

Bei Quellen mit Gedächtnis können wir ahnen, welches Zeichen als nächstes kommt,
daher sinkt der Informationsgehalt und die Redundanz nimmt zu.

\begin{tikzpicture}[->,>=stealth',shorten >=1pt,auto,node
                    distance=2.8cm,semithick]

	\node[state] (A)                    {$A$};
	\node[state] (B) [below left of=A]  {$B$};
	\node[state] (C) [below right of=A] {$C$};

	\path (A) edge [bend right] node [above left]  {$\sfrac{4}{5}$}  (B)
	          edge [bend left]  node               {$\sfrac{1}{5}$}  (C)
	      (B) edge [bend right] node [below right] {$\sfrac{1}{2}$}  (A)
	          edge [loop below] node               {$\sfrac{1}{2}$}  (B)
	      (C) edge [bend left]  node               {$\sfrac{1}{2}$}  (A)
	          edge [bend left]  node               {$\sfrac{2}{5}$}  (B)
	          edge [loop below] node               {$\sfrac{1}{10}$} (C)
	;

\end{tikzpicture}



\subsubsection*{1. Aufstellen der Matrix für die bedingte Wahrscheinlichkeit}

Die Werte können aus dem Diagramm ausgelesen werden.

\begin{tabular}[H]{|c|c|c|c|c|}
	\hline
	$p(y|x)$ & $Y=$ & $A$ & $B$ & $C$ \\
	\hline
	\multirow{3}{*}{$X=$} & $A$ geht über in & $0$ & $\frac{4}{5}$ & $\frac{1}{5}$ \\
	\cline{2-5}
	& $B$ geht über in & $\frac{1}{2}$ & $\frac{1}{2}$ & $0$ \\
	\cline{2-5}
	& $C$ geht über in & $\frac{1}{2}$ & $\frac{2}{5}$ & $\frac{1}{10}$ \\
	\hline
\end{tabular}


\subsubsection*{2. Berechnen der Wahrscheinlichkeit der einzelnen Zeichen}

Aufstellen der vier Gleichungen:

% TODO unterschied P vs p?
\begin{equation*}
	\begin{array}{rclclcl}
		1    &=& p(A)                   &+& p(B)                   &+& p(C) \\
		p(A) &=& p(A) \cdot 0           &+& p(B) \cdot \frac{1}{2} &+& p(C) \cdot \frac{1}{2} \\
		p(B) &=& p(A) \cdot \frac{4}{5} &+& p(B) \cdot \frac{1}{2} &+& p(C) \cdot \frac{2}{5} \\
		p(C) &=& p(A) \cdot \frac{1}{5} &+& p(B) \cdot 0           &+& p(C) \cdot \frac{1}{10}
	\end{array}
\end{equation*}

Daraus ergibt sich:
\begin{align*}
	p(A) &= \frac{1}{3} \\
	p(B) &= \frac{16}{27} \\
	p(C) &= \frac{2}{27}
\end{align*}


\subsubsection*{3. Zusammenfassen}

% TODO hä???
\begin{tabular}[H]{|c|c|c|c|c|}
	\hline
	$p(y|x)$ & $Y=$ & $A$ & $B$ & $C$ \\
	\hline
	\multirow{3}{*}{$X=$} & $A = \frac{1}{3}$ & $0 = 0$ & $\frac{4}{5} = \frac{4}{15}$ & $\frac{1}{5} = \frac{1}{15}$ \\
	\cline{2-5}
	& $B = \frac{16}{27}$ & $\frac{1}{2} = \frac{8}{27}$ & $\frac{1}{2} = \frac{8}{27}$ & $0 = 0$ \\
	\cline{2-5}
	& $C = \frac{2}{27}$ & $\frac{1}{2} = \frac{1}{27}$ & $\frac{2}{5} = \frac{4}{135}$ & $\frac{1}{10} = \frac{1}{135}$ \\
	\hline
\end{tabular}


\subsubsection*{4. Informationsgehalt der Zeichen}

Der Informationsgehalt berechnet sich wie folgt:
\[
	I(X_k) = \lb\left(\frac{1}{p(x_k)}\right)
\]

Daraus ergibt sich:
{%
	\renewcommand{\arraystretch}{2}
	\begin{equation*}
		\begin{array}{rclcl}
			I(A) &=& \lb\left(\frac{1}{\sfrac{1}{3}}\right) &=& 1.585 \, \textrm{Bit} \\
			I(B) &=& \lb\left(\frac{1}{\sfrac{16}{27}}\right) &=& 0.755 \, \textrm{Bit} \\
			I(C) &=& \lb\left(\frac{1}{\sfrac{2}{27}}\right) &=& 3.755 \, \textrm{Bit}
		\end{array}
	\end{equation*}
}%
