\section{Kompressionsverfahren}

Das Ziel der Kompression ist die Reduktion der Redundanz und das Entfernen der
Irrelevanz. Man unterscheidet drei Varianten:

\begin{itemize}
	\item Statische Verfahren\\
		- z.B. Huffman-Codierung für bekannte Daten (beispielsweise die Deutsche
		Sprache)\\
		- Blockdaten
	\item Adaptive Verfahren\\
		- z.B. Huffman-Codierung mit gemessener Häufigkeitsverteilung\\
		- Blockdaten
	\item Dynamische Verfahren\\
		- z.B. LZ77 (Lempel-Ziv)\\
		- Streamingdaten
\end{itemize}

\subsection{Huffman}

TODO

\subsection{Lempel-Ziv}

Das Lempel-Ziv Verfahren ist ein Dynamisches Verfahren zwecks Ausnutzung
wiederkehrender Muster. Es wird nicht der Code sondern die codierten Phrasen
übertragen und in einem Phrasenspeicher (``Wörterbuch'') gespeichert. Als
Phrasenspeicher wird ein ständig wachsender Baum erzeugt, mit den Knoten als
Referenzen. Treten die selben Phrasen wiederholt auf, wird nicht die Phrase
sondern nur der Knoten des Phrasenspeichers referenziert.

Funktionsweise:

\begin{enumerate}
	\item Die zu codierende Zeichenfolge wird in Teilfolgen aufgeteilt, welche
		alle verschieden sind.
	\item Bei den zu übertragenden Zeichen werden die Position im Baum und evtl.
		neue Zeichenfolgen angegeben. Beispiel: $(3,1)$ $\rightarrow$ Position 3 im
		Baum, neues Zeichen 1.
	\item Es wird ein Phrasenspeicher (Decoderbaum) aufgebaut, welcher bereits
		übertragene Zeichenfolgen speichert.
\end{enumerate}

Voraussetzung für gute Kompression ist, dass die Zeichenfolge viele
Regelmässigkeiten besitzt. Ansonsten kann es sein, dass der ``komprimierte
Code'' grösser als der Ursprüngliche ist.

Probleme:
\begin{itemize}
	\item Effizientes Suchen und Einfügen in den Baum
	\item Grösse des Baumes -- der Baum kann nur bis zu einer gewissen Grösse wachsen
\end{itemize}

Siehe \autoref{example:lempel-ziv} auf Seite~\pageref{example:lempel-ziv}
für ein Anwendungsbeispiel.
