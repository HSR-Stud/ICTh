\section{Kompressionsverfahren}

\subsection{Huffman}

TODO

\subsection{Lempel-Ziv}

Funktionsweise:

\begin{enumerate}
	\item Die zu codierende Zeichenfolge wird in Teilfolgen aufgeteilt, welche
		alle verschieden sind.
	\item Bei den zu übertragenden Zeichen werden die Position im Baum und evtl.
		neue Zeichenfolgen angegeben. Beispiel: $(3,1)$ $\rightarrow$ Position 3 im
		Baum, neues Zeichen 1.
	\item Es wird ein Phrasenspeicher (Decoderbaum) aufgebaut, welcher bereits
		übertragene Zeichenfolgen speichert.
\end{enumerate}

Voraussetzung für gute Kompression ist, dass die Zeichenfolge viele
Regelmässigkeiten besitzt. Ansonsten kann es sein, dass der ``komprimierte
Code'' grösser als der Ursprüngliche ist.

Probleme:
\begin{itemize}
	\item Effizientes Suchen und Einfügen in den Baum
	\item Grösse des Baumes -- der Baum kann nur bis zu einer gewissen Grösse wachsen
\end{itemize}

Siehe \autoref{example:lempel-ziv} auf Seite~\pageref{example:lempel-ziv}
für ein Anwendungsbeispiel.
