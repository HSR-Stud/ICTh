\section{Verschlüsselungsverfahren}
\subsection{RSA}

Um eine Nachricht zu verschlüsseln benotigt man zwei grosse Primzahlen p und q.
$p \cdot q = N$ und eine beliebige Zahl $e$ zwischen 1 und $N-1$ und teilerfremd
zu $\Phi(N)$. Die Nachricht ist ein Zahl $M$. Um diese zu verschlüsseln rechnet
man $M^e \bmod N = C$. $C$ ist der verschlüsselte Text. $N$ und $e$ sind der
public key.

Um die Nachricht $C$ zu entschlüsseln rechnet man $C^d \bmod N = M$. $d$ ist der
Private Key.
\begin{align*}
	C^d \bmod N &= M \\
	C^d & = M
\end{align*}
\begin{align*}
	C^d \bmod N  =  (M^e)^d \bmod N = M^{ed} \bmod N
\end{align*}

Satz von Euler:
\begin{align*}
m^{\Phi (n)} \bmod n = 1
\end{align*}

$M^{ed}$ ist dasselbe wie $M^{ed \bmod \Phi (pq)} \bmod pq$. Wenn man $d$ so
wählt, dass es das inverse Element von $e$ ist, dann gilt $e \cdot d \bmod
\Phi(pq) = 1$. Damit rechnet man $M^1$ sobald man $M^e$ mit dem inversen Element
$d$ potenziert.

\textbf{Beispiel:}
Mit zwei Primzahlen und der Zahl $e$ kann man ein Schlüsselpaar bilden. Mit dem
Paar $(e,N)$ kann man eine Zahl verschlüsseln. Mit $(d,N)$ kann man sie
entschlüsseln.
\begin{description}
\item[Gegeben:] Primzahlen $p=11$ und $q=7$ und die Zahl $e$
\item[Gesucht: ] Schlüsselpaare für RSA
\end{description}
Zuerst muss $N$ berechnet werden:
\[
N = p*q
\]
\[
77 = 11*7
\]
Damit hat man schon ein Teil des Schlüssels $(e,N)$. Jetzt muss man noch die
Zahl $d$ für den zweiten Schlüssel $(d,N)$ berechnen. Für $d$ muss gelten $e*d
mod \Phi(N) = 1$. Für die weiteren Schritte wird die Zahl $\Phi(N)$ benötigt.
\[
\Phi(N) = (p-1) * (q-1)
\]
\[
60 = (11-1) * (7-1)
\]
Um die Zahl $d$ für den zweiten Schlüssel herauszufinden muss man das ggT von
$\Phi(N)$ und $e$ berechnen. Das ist logischerweise 1. Man möchte aber nicht das
ggT herausfinden, sondern man benötigt die Gleichungen des euklidischen
Algorithmus. Zuerst multipliziert man $e$ damit das Produkt $\leq \Phi(N)$ ist
und addiert die Differenz $n$
\[
\Phi(N) = k * e + n
\]
Bei der nächsten Gleichung setzt man für $\Phi(N)$ $e$ ein und für $e$ $n$.
Irgendwann wird $n=0$ und $e=1$. Das wäre eigentlich das ggT. Das haben wir zwar
schon vorher gewusst. Man benötigt aber die Zwischenschrite. Mit Zahlen:
\begin{eqnarray}
\label{eqn:inv3}
60 & = & 8 * 7 + 4 \\
\label{eqn:inv2}
7 & = & 1 * 4 + 3\\
\label{eqn:inv1}
4 & = & 1 * 3 + 1\\
3 & = & 3 * 1 + 0
\end{eqnarray}
Die letzte Gleichung braucht man eigentlich nicht. Die letzte Gleichung
\ref{eqn:inv1} kann geschrieben werden als:
\begin{equation}
\label{eqn:subs}
1 = 4 - 1*3
\end{equation}
Gleichung \ref{eqn:inv2} kann als
\[
3 = 7 - 1*4
\]
geschrieben werden.
Mit \ref{eqn:inv2} kann die Zahl 3 in \ref{eqn:subs} ersetzt werden.  
\begin{equation}
1 = 4 - 1*(7-1*4)
\end{equation}
Diese Gleichung muss man so umformen, dass die Zahl 4 ein Faktor des zweiten
Summand ist, damit man sie danach wieder substituieren kann.
\begin{align}
1 &= 4 - 7+4 \\
1 &= -7 + 2*4
\end{align}
Die Zahl 4 substituiert mit Gleichung \ref{eqn:inv3} substituiert ergibt
\begin{equation}
1 = -7 + 2*(60-8*7)
\end{equation}
Diese Gleichung muss jetzt so umformen dass man etwas in der Form
\begin{equation*}
1 = 60*k + 7*d
\end{equation*}
oder
\begin{equation*}
1 = \Phi*k + e*d
\end{equation*}
erhält. Weil damit ist die Bedingung $e*d \bmod \Phi(N) = 1$ erfüllt.
\begin{eqnarray}
1 & = & -7 + 2*60-16*7 \\
1 & = & 2 * 60-16*7-7 \\
1 & = & 2 * 60-16*7-1*7\\
1 & = & 2 * 60-7(16+1)\\
1 & = & 2 * 60-17*7
\end{eqnarray}
Wenn $d$ negativ ist muss man noch $d \bmod N$ rechnen. Also $\unaryminus 17
\bmod 60 = 43$.  Somit ist $d = 43$
