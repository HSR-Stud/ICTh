\section{Verschlüsselungsverfahren}

\subsection{RSA}

RSA ist ein Public-Key-Verschlüsselungsverfahren. Mit dem öffentlichen Public
Key werden Nachrichten für den Empfänger verschlüsselt und mit dem geheimen
Private Key kann die Nachricht wieder entschlüsselt werden.

Ein Anwendungsbeispiel findet sich im \autoref{example:rsa} auf
Seite~\pageref{example:rsa}.

\subsubsection*{Schlüsselpaar berechnen}

Um eine Nachricht zu verschlüsseln benötigt man zwei grosse Primzahlen $p$ und
$q$. Aus diesen zwei Primzahlen bildet man das Produkt $N$.
\begin{equation}
	\label{eqn:rsa:N}
	N = p \cdot q
\end{equation}
Als nächstes braucht man eine beliebige Zahl $e$ (encode) kleiner als $N$ und
teilerfremd\footnote{Zwei natürliche Zahlen sind teilerfremd oder relativ prim,
wenn sie keine gemeinsamen Primfaktoren aufweisen.} zu $\phi(N)$. Mithilfe
dieser Zahl wird $d$ (decode) berechnet, so dass die folgende Gleichung erfüllt
ist:
\begin{equation}
	\label{eqn:rsa:d}
	(e \cdot d) \bmod \phi(N) = 1
\end{equation}
$\phi$ bezeichnet dabei die Eulersche Phi-Funktion. Sie gibt für jede natürliche
Zahl $n$ an, wie viele positive ganze Zahlen es gibt, welche teilerfremd zu $n$
und kleiner als $n$ sind. Die Berechnung von $\phi(n)$ ist bei einem grossen $n$
und unbekannten Primfaktoren extrem zeitintensiv. Da in diesem Fall die zwei
Primfaktoren $p$ und $q$ jedoch bekannt sind, kann $\phi(N)$ folgendermassen
berechnet werden:
\begin{equation}
	\phi(N) = (p-1)(q-1)
\end{equation}
Nun kann man \eqref{eqn:rsa:d} mithilfe des Erweiterten Euklidischen Algorithmus
(siehe nachfolgendes Beispiel) lösen und erhält somit $(e,N)$ als \textbf{Public
Key} und $(d,N)$ als \textbf{Private Key}. Die restlichen Zwischenresultate
werden ab nun nicht mehr benötigt.

\subsubsection*{Ver- und Entschlüsselung}

Eine Nachricht $M$ kann mit dem Public Key folgendermassen verschlüsselt werden:
\begin{equation}
	M^e \bmod N = C
\end{equation}
Um die verschlüsselte Nachricht $C$ zu entschlüsseln, wird der Private Key
benutzt:
\begin{equation}
	C^d \bmod N = M
\end{equation}
